\section*{Resumen}

\normalsize
El resumen consiste en la presentación clara y concisa de los puntos más relevantes del
trabajo de manera de entregar una idea general del documento. El resumen antecede la
introducción y en los trabajos de título no debe superar las 350 palabras. El contenido del
resumen debe estar constituido por una secuencia de oraciones compuestas y no por una
enumeración de tópicos. El primer párrafo debe presentar el problema principal a abordar. El
segundo párrafo debe explicar la solución desarrollada. El tercer y último párrafo debe
presentar los resultados y conclusiones del trabajo. No deben incluirse fórmulas matemáticas
ni figuras. Después del resumen se deben incluir las palabras claves del documento. 

\noindent
\textit{Palabras Clave: lenguajes, heurísticas, agentes, patrones de diseño.}

\noindent
\stdsection*{Abstract}

\normalsize
The abstract consists in a clear and concise presentation of the most important points of
the work in order to give an overview of the document. The abstract precedes the introduction
and in the “trabajos de título” it must not exceed 350 words. The abstract content must be
constituted by a sequence of compound sentences and not by an enumeration of topics. The
first paragraph introduces the main problem to tackle. The second paragraph explains the
developed solution. The third and last paragraph presents the results and the conclusions of the
work. Mathematical formulae and figures must not be included. After the abstract, the
keywords of the document must be included.

\noindent
\textit{Keywords: languages, heuristics, agents, design patterns. }
