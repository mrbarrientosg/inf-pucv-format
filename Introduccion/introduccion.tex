\section{Introducción}

La forma en que se escribe y presenta un documento puede incidir de modo importante
en la lectura y comprensión que otros hagan de él. Es por este motivo que la Escuela de
Ingeniería Informática de la Pontificia Universidad Católica de Valparaíso ha hecho esfuerzos
permanentes en la búsqueda de un formato adecuado para utilizar por parte de los estudiantes
en la preparación de cualquier documento académico que deban presentar en sus asignaturas.

El objetivo de este nuevo formato es estandarizar aspectos generales que sean aplicables
a documentos en distintos contextos, tales como trabajos de asignaturas, informes académicos,
trabajos de título, etc. Por este motivo, su uso es obligatorio para todos los informes
académicos elaborados en la Unidad Académica.

El documento comienza con una descripción de la presentación gráfica de los
documentos académicos. Luego se describe la estructura y contenido de la portada.
Posteriormente se describe con más detalle, cómo confeccionar el resumen y el abstract.
Asimismo, se plantea el modo en que se deben numerar lo capítulos y secciones, el tipo de
letra y ubicación de la numeración de páginas. Otro aspecto importante es la descripción de
cómo y cuándo realizar citas. A continuación se describe de qué manera se deben presentar las
figuras y tablas en el texto. Finalmente se presentan como anexos algunos ejemplos de
portadas, glosario de términos, lista de abreviaturas o siglas, simbología, lista de figuras y lista
de tablas.

La Escuela espera que este formato permita mejorar la presentación de los documentos
preparados por sus estudiantes. 